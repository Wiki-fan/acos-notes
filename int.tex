\documentclass[main]{subfiles}

\begin{document}

Когда поступает прерывание, в стек отправляется адрес возврата (IP),
регистр флагов, сегментный регистр (один или несколько: код, стек) --- это
нужно из соображений, что происходит переход к обработчику прерываний,
который может не иметь отношения к исполняемому коду, малое упрятывание.
Кроме того, происходит маскирование прерываний, старая маска кладётся в стек
(нужно, чтобы обработчик прерываний не был прерван другим прерыванием).
После этого происходит большое упрятывание.

У каждого прерывания есть определённый номер, аппаратура обращается в таблицу,
которая называется вектором прерываний (по адресу из регистра idtr на x86) и
содержит точки входа в обработчики прерываний. По номеру прерывания в ней
ищется нужный обработчик и осуществляется переход на него. Обработка прерываний ---
это дальний переход с точки зрения ассемблера. Обработчик прерываний должен
все необходимые регистры тоже сложить в стек и разрешить важные прерывания.
После этого происходит, собственно, обработка прерываний, и вызывается команда
iret, которая восстановит нормальный ход работы.

Прерывания с меньшим номером более приоритетны, это учитывается при одновременном
поступлении нескольких прерываний.
Вторая интересная особенность заключается в том, что (A)PIC позволяет
переназначать номера прерываний, тем самым устанавливая приоритет.

Регистр прерываний содержит номер прерывания, он может являться
многоуровневым, это же касается регистра маски прерывания.
С одной стороны, обработка прерываний --- это работа ядра ОС, с другой стороны,
обычно обработчики прерываний просто передают управление драйверам устройств
(если это прерывание устройства).
Что вообще есть драйвер? Это некоторая программа, которая предоставляет
интерфейс работы с устройством, обычно обработка соответствующих прерываний
также ложится на их плечи. В итоге, обработка прерываний в современных ОС
проходит на нескольких уровнях: вызывается изначальный обработчик в ядре ОС,
который находит нужные обработчики (обычно драйвера) и вызывает их.
\end{document}
