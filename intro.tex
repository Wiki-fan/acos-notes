\documentclass[main]{subfiles}

\begin{document}

\chapter{Введение}
%Алексей Николаевич Сальников alexey.salnikov@gmail.com + Hangouts
%salnikov@cs.msu.ru
%salnikov@jabber.ru
%vk.com/asalnikov
Краткий план курса:
\begin{enumerate}
\item Всё про оборудование: принципы работы, интерфейсы для программ
\item Внутреннее устройство ОС
\item Программные интерфейсы к ОС: POSIX, системы вызовов
\end{enumerate}

%-Wall -ansi -pedantic -g

Причины использования C:
\begin{enumerate}
\item Простота
\item Ручное управление памятью
\item Свобода программиста
\end{enumerate}

Что почитать:
\begin{enumerate}
\item Google, StackOverflow, Linux.org.ru, wiki.osdev.org
\item Андрей Робичевский
\item Бовет, Чезати "Understanding Linux Kernel"
\item Рочкинд "Advanced Unix programming"
\end{enumerate}

\chapter{Общее устройство компьютеров}
Устройства, с которыми взаимодействует пользователь, содержат в себе
несколько уровней ПО:
\begin{enumerate}
\item Прикладной уровень --- ПО, с которым непосредственно работает пользователь
\item Уровень системного программирования --- библиотеки, компиляторы, отладчики
\item Уровень управления логическими ресурсами --- интерфейс, позволяющий абстрагироваться
от деталей аппаратной платформы; виртуальная память, понятие файла, программы, сетевого
интерфейса; libc, POSIX
\item Уровень управления физическими ресурсами --- ядро, драйверы
\item Аппаратура
\end{enumerate}

А что, собственно, такое компьютер? Сейчас доминируют IBM-совместимые ПК, которые
содержат следующие компоненты:
\begin{enumerate}
\item Системная (материнская) плата: шины, контроллеры (в одном чипе --- мост),
разъёмы подключения устройств и питания, BIOS/UEFI (энергонезависимая память),
тактовый генератор, часы
\item Процессор
\item Оперативная память
\end{enumerate}

\section{Шина}
Обыкновенно представляет медное напыление на диэлектрик в совокупности с
разъёмами и контроллерами, иногда для повышения
быстродействия используют оптоволокно. Основными характеристиками
можно назвать следующие:
\begin{itemize}
\item Частота работы
\item Ширина шины
\item Пропускная способность
\item Количество абонентов
\end{itemize}
Поскольку к шине подключено несколько устройств, могут возникнуть конфликты,
если несколько устройств-абонентов пытаются одновременно воспользоваться шиной;
их разрешает контроллер, называемый арбитром шины.

\section{Память}
В современном виде представляет собой набор запоминающих элементов,
объединённых в одну плату.

Основные характеристики:
\begin{itemize}
\item Частота
\begin{itemize}
\item Время цикла (тайминг): как часто можно отправлять запросы на запись / на чтение
\item Время чтения / записи ячейки: сколько времени занимает одна операция
\item Время рецикла: время, через которе необходимо перезаписать данные для
их сохранения
\end{itemize}
\item Частота работы чипа памяти
\end{itemize}

\section{Процессор}
Процессор представляет собой микросхему с набором контактов-ножек. Такт процессора ---
минимальный отрезок времени, который гарантирует фиксированное значение напряжений на этих ножках.

Основные характеристики:
\begin{itemize}
\item Тактовая частота
\item Набор инструкций
\item Разрядность
\item Количество ядер
\item Кэш (объём, скорость)
\item Глубина конвейеров
\item Тепловыделение
\end{itemize}

\end{document}
