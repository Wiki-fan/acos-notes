\documentclass[main]{subfiles}

\begin{document}
Возвращаемся к вопросу, что такое ассемблерная программа.
Ассемблерная программа --- это текстовый файл с расширением .s (.asm в Windows).
В этом файле содержатся команды в некотором синтаксисе, мы будем
работать с синтаксисом AT\&T. Его генерирует gcc и соответствующий ему
ассемблер --- это gas. Вот основные соглашения в AT\&T синтаксисе: 
\begin{itemize}
\item Все регистры именуются \%ИМЯ\_РЕГИСТРА
\item Всё, что начинается с доллара --- числа (такая же запись, как в C: десятичная
\$10, шестнадцатеричная \$OxFF и восьмеричная \$011)
\item Просто текст --- имя команды
\item Служебные слова начинаются с точки (макросы, управление компиляцией)
\item Результат записывается в последний аргумент
\end{itemize}

Программа разбита на секции:
\begin{description}
\item[.text] --- команды
\item[.rodata] (обычно примыкает к секции .text) --- константы: их
можно объявлять и устанавливать им значение, но их нельзя изменять
\item[.data] --- переменные, которыми будет пользоваться программа
\item[.bss] --- статические переменные. Память под них не отводится в объектном файле,
но компилятор/линкер должен знать размер этой секции, который здесь и прописывается.
\end{description}

NB: для 64-битных чисел используется суффикс u.

.int --- резервирует четыре байта там, где она встретилась.
Можно добавить число, тогда это --- количество участков.  Есть
также .int, .long, .short, .byte. Присутствует понятие строк:
.string "строка". Как описать переменную? Во-первых, выделить память для неё,
во-вторых, нужно дать ей имя. Для этого в ассемблере используют метки,
их можно использовать как адрес для перехода или как аргумент.
Чтобы задать метку, нужно написать ИМЯ\_МЕТКИ: в начале строки.

a: .int 50 --- выделит 4 * 50 байт.
Для работы с ними нужно использовать круглые скобки:
movl \%eax, (a) переместит значение из регистра в первый элемент массива a.
movl \%eax, +40(a) --- в сорок первый элемент.

Пример простейшей программы (инклюды опущены):
\begin{lstlisting}[numbers=left]
int func(long int a, long int b, long int c,
	 long int d, long int e, long int f) {
  int result;
  result = a + b + c + d +e +f;
  printf("result = %d\n", result);
  return 10;
}

int main() {
  func(1, 2, 3, 4, 5, 6);
  return 0;
}
\end{lstlisting}

В результате компиляции без каких-либо ключей, мы получим такой код на
языке ассемблера:
\lstinputlisting[multicols=2,numbers=left]{asm-example.s}

В данном случае, происходит вообще бессмысленная деятельность: из-за несогласованности
частей компилятора, данные, переданные через регистры, перемещаются на стек.
Для этого также стек увеличивается на 64 байта.

В тексте программы был упомянут символ printf. Откуда же он берётся?
Этим занимается линковщик.
Статическая линковка: есть набор объектных файлов, они предоставляют и требуют
некоторые символы. Линковщик создаёт новый файл, добавляет в его начало
некоторый заголовок (зависит от ОС) и собирает секции .text из всех
файлов в одну, аналогично --- для .data. Статическая библиотека (*.a в Linux ---
архив секций, *.lib в Windows) таким же образом добавляется сюда же.
Далее линковщик подставляет вместо символов конкретные адреса относительно
начала файла. Также выполняются некоторые другие действия.
Например, prog зависит от a, а a зависит от b. Что примечательно,
требуется линковать их в определённом порядке: для оптимизации ненужные
символы отбрасываются, поэтому правильной последовательностью линковки
будет prog, a, b.

При статической линковке программа сильно растёт в размере, а некоторые
библиотеки используются более, чем одной программой. Для экономии памяти
используется динамическая линковка: в тело программы включается код
линковщика (ld.so для Linux), но как узнать адрес функции, ведь он может
зависеть от запуска? Для этого в секции .data заводится таблица динамических
функций (заполняется при запуске), и перед вызовом call эта функция ищется
в этой таблице. В gcc динамическая линковка включается ключом -shared,
ключ -pic говорит, что нужно генерировать позиционно-независимый код, т. к. 
библиотека может лежать по разным адресам оперативной памяти.
\end{document}
